\documentclass[a4paper]{article}

\usepackage[utf8]{inputenc}

\usepackage[parfill]{parskip}

\setlength{\parindent}{0pt}

\begin{document}

\textbf{Zeit:} 11.11.13, 15:50-16:30

\textbf{Anwesend:} Lukas, Felix, Daniel, David, Jonas (Protokollant)

\textbf{Nächstes Treffen:} 14.11.13, 14 Uhr Technologiefabrik

\section{Inhalt}

Es wurden die eingereichten Abschnitte des Pflichtenhefts besprochen und Änderungen vorgeschlagen.

\section{Änderungen des Pflichtenhefts}

\begin{itemize}
	\item Abgrenzung: Hardwarebeschleunigung und Audio (Daniel)
	\item Produktumgebung: Hardware entfernen oder überdenken (Jonas)
	\item Funktionale Anforderungen: Audio entfernen. "Testdaten" präzisieren (Lukas)
	\item Produktdaten: Modifikatoren entfernen. Graph/Layout sowie Testdaten hinzufügen (David)
	\item Spezielle Anforderungen an die IDE: Qt 5.2 und Doxygen hinzufügen (David)
	\item Produktleistungen: Vorsicht, nicht endgültig!
\end{itemize}

\section{Aufgaben bis Mittwoch 13.11.2013}

\begin{itemize}
	\item UML-Tool suchen (Jonas)
	\item Qt-Quick ausprobieren (Lukas)
	\item Unittests (Felix)
	\item Kontakttabelle verschicken (David)
	\item Vorläufige Version des Pflichtenhefts am Mittwoch, den 13.11. um 19.00 an Sebastian schicken (Daniel)
\end{itemize}
\end{document}
