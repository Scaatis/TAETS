\section{Funktionale Anforderungen}

\subsection{Entwurf des Testlaufs}

\begin{speclist}[F]
\setcounter{specnum}{11010}

\spec Graph-basierter Entwurf \\
Der Benutzer soll einen Testlauf durch einen azyklischen Graphen definieren können. Er verbindet dafür mehrere Knoten miteinander.

\spec Knoten \\
Ein Knoten ist eine kleine Recheneinheit, die Eingabeparameter verrechnet und Ausgabe produziert. Ein Knoten hat einen oder mehrere Ein- bzw. Ausgabestecker.

\spec Graphstecker \\
Ein Graphstecker ist der Teil der Schnittstelle eines Knoten, die Daten aufnimmt bzw. Daten ausgibt. Graphstecker haben einen festen Typ.

\spec Typen von Graphsteckern
\begin{itemize}
	\item Fließkommazahl
	\item YUV-Bild
\end{itemize}

\spec Verbindung zwischen Graphsteckern \\
Graphstecker können miteinander verbunden werden. Eine Verbindung kann nur zwischen Steckern gleichen Typs stattfinden.

\end{speclist}

\subsection{Knoten des Graphen}

\subsubsection{Eingabeknoten}
Eingabeknoten haben nur Zahlen als Eingabestecker und Bilder als Ausgabestecker.

\begin{speclist}[F]
\setcounter{specnum}{21010}

\spec YUV-Video Streams \\
Es können unkomprimierte YUV-Video Streams gelesen werden. Der Start-Frame ("`Offset"') und die Anzahl der ausgelesenen Frames ("`Länge"') sind statisch (pro Testdurchlauf, also keine Eingabestecker) veränderbar.

\spec Generatoren \\
Es können YUV-Bilder prozedual generiert werden. Dazu stehen mehrere Algorithmen zur Verfügung:
\begin{itemize}
	\item Einfarbig
	\item Rauschen
\end{itemize}
Die Eingabeparameter der Generatoren können durch den Graph variiert werden.

\optspec Einlesen von verbreiteten Videoformaten \\
Es können weit verbreitete Videoformate, wie MPEG oder H.264 eingelesen werden. Auch hier können der Offset und die Länge variiert werden.

\optspec Einlesen von Videos mit anderer Framerate \\
Es können auch Videos mit anderen Framerates als der des Graphen gelesen werden. Diese werden vor der Ausgabe umgerechnet.

\end{speclist}

\subsubsection{Modifikatoren}
Modifikatoren verändern Bilder. Sie machen Vor- bzw. Nachberechnungen der Testdaten.

\begin{speclist}[F]
\setcounter{specnum}{22010}

\spec Helligkeit und Kontrast \\
Grundlegender Modifikator mit zwei Eingabezahlen.

\spec Grundlegende Bildkombination \\
Bildkombinationen nehmen zwei Eingabebilder und eine Zahl und geben ein kombiniertes Bild aus. Es stehen folgende grundlegende Bildkombinationsmethoden zur Verfügung:
\begin{itemize}
	\item Addition
	\item Subtraktion
	\item Mix
\end{itemize}

\spec Scharfzeichner \\
Ein Knoten zum Anwenden eines Scharfzeichenalgorithmus auf ein Bild. Folgende Algorithmen stehen zur Verfügung:
\begin{itemize}
	\item Unscharfmaskierung (https://de.wikipedia.org/wiki/Unscharfmaskierung)
	\item Laplace-Filter (https://de.wikipedia.org/wiki/Laplace-Filter)
\end{itemize}

\spec Weichzeichner \\
Ein Knoten zum Anwenden eines Weichzeichenalgorithmus auf ein Bild. Folgende Algorithmen stehen zur Verfügung:
\begin{itemize}
	\item Lineares Weichzeichnes
	\item Gaussches Weichzeichnen
\end{itemize}

\optspec Verkleinern \\
Folgende Verkleinerungsalgorithmen stehen zur Verfügung:
\begin{itemize}
	\item Nächster Pixel
	\item Lineare Interpolation
\end{itemize}

\optspec Vergrößern \\
Folgende einfache Vergrößerungsalgorithmen stehen zur Verfügung:
\begin{itemize}
	\item Nächster Pixel
	\item Lineare Interpolation
\end{itemize}

\optspec Resampling \\
Komplexer Vergrößerungsalgorithms mit besseren Ergebnissen.

\optspec Zeit-Variierte Zahlen \\
Knoten, der Zahlenwerte generiert, die sich im Laufe der Videozeit ändern. Folgende Methoden stehen zur verfügung:
\begin{itemize}
	\item Linear
	\item Mathematische Funktionen (sinus, log, ...)
	\item eventuell Bezier Kurve
\end{itemize}

\optspec Matlab Integration \\
Knoten, der Matlab-Programme kapselt und die Implementation eigener Modifikatoren erlaubt. Könnte auch der Basisknoten für fast alle anderen Modifikatoren sein.

\end{speclist}

\paragraph{Encoder-Modifikator}

\begin{speclist}[F]

\spec Encoderknoten \\
Der Encoderknoten encodiert das eingehende Bild und gibt eine codierte Version aus. Hier findet der eigentliche Encodingprozess mit dem zu testenden Videoencoder statt.

\optspec Metadaten während des Encodings \\
Der Encoderknoten macht zusätzliche Ausgaben, die Metadaten während des Encodingprozesses darstellen. Diese sind:
\begin{itemize}
	\item Rechenzeit pro Frame
	\item Bitrate
	\item Codierungsmethoden der Blöcke
\end{itemize}

\end{speclist}

\subsubsection{Ausgabeknoten}
Es steht ein Ausgabeknoten zur Verfügung, der "`Inspektor"'.

\begin{speclist}[F]
\setcounter{specnum}{23010}

\spec Inspektor \\
Der Inspektorknoten greift Testdaten während der Berechnung ab und speichert diese für die spätere Auswertung.

\end{speclist}

\subsection{Analyse der Testdaten}

\begin{speclist}[F]
\setcounter{specnum}{31010}

\spec Dialog für Inspektorknoten \\
Nach einer Berechnung kann ein Dialog zu einem Inspektorknoten geöffnet werden. Hier werden die aufgezeichneten Datenreihen visualisiert und können analysiert werden.

\spec Visualisierung von Zahlen \\
Aufgezeichnete Zahlenwerte können pro Bild angezeigt werden.

\spec Visualisierung von Zahlenreihen \\
Die aufgezeichneten Zahlenreihen können in einem Diagramm angezeigt werden, um den zeitlichen Verlauf anzusehen. Es soll möglich sein, mehrere Zahlenreihen in ein einziges Diagramm zu laden.

\spec Visualisierung von Bildern \\
Einzelne berechnete Bilder können angezeigt werden.

\spec Zusätzliche Bildinformationen \\
Zu angezeigten Bildern können zusätzliche Informationen dargestellt werden:
\begin{itemize}
	\item Histogram
	\item Farbraum
\end{itemize}

\spec Vergleich von Bildern \\
Wenn der Inspektor mehrere Bilderreihen aufgenommen hat, sollen je zwei miteinander verglichen werden können. Folgende Methoden können verwendet werden:
\begin{itemize}
	\item "`Side-by-Side"' Anzeige
	\item "`Scroll"'-Vergleich
	\item Differenzbild
	\item PSNR-Wert
	\item Überlagerung Histogramme
	\item Überlagerung Farbräume
\end{itemize}

\spec Visualisierung von Bildvergleichen \\
Folgende Werte aus dem Bildvergleich können in einem zeitlichen Diagramm dargestellt werden:
\begin{itemize}
	\item PSNR-Wert
	\item Differenz der Histogramme
	\item (Differenz der Farbräume)
\end{itemize}

\spec Vergleich von mehr als zwei Bilderreihen \\
Folgende Methoden können verwendet werden, um mehr als zwei Bilderreihen miteinander zu vergleichen:
\begin{itemize}
	\item "`Side-by-Side"' Anzeige
	\item "`Scroll"'-Vergleich
	\item Überlagerung Histogramme
	\item Überlagerung Farbräume
\end{itemize}

\spec Anzeige von Bilderreihen als Video \\
Die eingegangenen Bilderreihen sollen als Video abspielbar sein.

\end{speclist}

\subsection{Benutzeroberfläche}

\begin{speclist}[F]
\setcounter{specnum}{41010}

\spec Graphübersicht \\
Der Benutzer sieht eine grafische Übersicht des Graphen. Es werden Knoten, sowie deren Verbindungen angezeigt.

\spec Knoten hinzufügen \\
Der Benutzer kann neue Knoten hinzufügen.

\spec Knoten entfernen \\
Der Benutzer kann existierende Knoten entfernen. Abhängende Verbindungen werden auch entfernt.

\spec Knoten verbinden \\
Der Benutzer kann Knoten verbinden. Es wird ein Fehler angezeigt, wenn die Typen der Stecker nicht übereinstimmen.

\spec Knoten einstellen \\
Jeder Knoten hat spezielle Einstellungen. Diese können vom Benutzer verändert werden.

\spec Projekt speichern \\
Das Layout des Graphen, sowie alle Knoteneinstellungen und -verbindungen können in einer Projektdatei gespeichert werden.

\spec Projekt laden \\
Eine gespeicherte Projektdatei kann geladen werden.

\spec Testlauf berechnen \\
Es gibt einen Knopf, mit dem der Benutzer den Graphen berechnen kann. Zwischenergebnisse werden für Inspektoren temporär gespeichert.

\spec Testlauf Ergebnisse speichern \\
Die Ergebnisse eines berechneten Testlauf können für eine spätere Auswertung abgespeichert werden.

\spec Testlauf Ergebnisse laden \\
Zu einem Graphen dazugehörige Berechnungsergebnisse können geladen werden. Danach ist der Status des Programms der eines berechneten Testlaufes.

\spec Rechenzeit anzeigen \\
Beim Berechnen des Testlaufs wird für jeden Knoten aufgezeichnet, wie lange insgesamt gerechnet hat. Diese Rechenzeit kann sich der Benutzer anzeigen lassen.

\optspec Parallele Berechnung \\
Die Berechnung ist mehrfädig für maximale Geschwindigkeit.

\optspec Hardwaregestützte Berechnungen \\
Ausgewählte Knoten werden auf spezialisierter Hardware (Grafikkarte) berechnet.

\end{speclist}
