\section{Zielbestimmung}

\subsection{Musskriterien}

\begin{itemize}
	\item Eingabe: 
	\begin{itemize}
		\item Einlesen von .yuv Dateien
		\item Verwenden einer Farbe, eines Bildes
		\item generiertes Noise
	\end{itemize}
	\item Modifikatoren:
	\begin{itemize}
		\item Encoder
		\item Kombinieren / Differenz von Videos
		\item Weich- und Scharfzeichner
		\item Helligkeit und Kontrast
	\item Bewerten:
	\begin{itemize}
		\item Verwenden von Inspektoren (Knoten)
		\item Generierung eines Zeitdiagramms
		\item Messung der Rechenzeit
		\item Berechnen des PSNR
		\item Berechnung des Histograms / Farbraums einzelner Frames des Videos
	\end{itemize}
	\item Ausgabe: Das ausgegebene Video kann gespeichert werden
\end{itemize}

\subsection{Sollkriterien}

\begin{itemize}
	\item Unterstützung anderer Kodierer
	\item Integration mit Matlab
	\item Metadaten auslesen (sowohl "Standart"Metadaten, wie auch welche, die nur von manchen ausgegeben werden)
	\item Weitere Modifikatoren: Skalieren des Videos, Resampling
	\item Implementierung von Zeitverschiebungsparametern
\end{itemize}

\subsection{Kannkriterien}

\begin{itemize}
	\item Unterstützung unterschiedlicher Framerates 
	\item Hardwarebeschleunigung (z.B. CUDA)
	\item Implementierung von Multithreading
	\item Audiospur des Videos abspielen
\end{itemize}

\subsection{Abgrenzungskriterien}

\begin{itemize}
	\item Es ist für erweiterte Benutzer gedacht
\end{itemize}